\subsection{SAR logic reset}
During sampling of the ADC input the SAR logic stage is reset ($CK=1$). In the
enable logic of the first  stage $EI=\overline{CK}=0$, and node $A=1$,
while $EO=0$. Thus $EI=EO=0$ of all subsequent stages. The CDAC
state control has $D_{P0}=D_{N1}=0$ while $D_{N0}=D_{P1}=1$. In the clock generation
loop of the first stage $CI=0$, and since node $B=0$, then
$CO=0$. Accordingly $CI=CO=0$ for all subsequent stages.

\subsection{Comparator decision}
The first comparator decision is initiated by $CK=0$ (steps 1 and 2 in Fig. \ref{fig_timing}), at that point the
first latch ($M_{N0}-M_{N2}, M_{P0}$) in the enable logic is armed, and as soon as $(P||N)=1$,
then $A=0$. This arms the second latch ($M_{N3},M_{P1}-M_{P3}$) in the enable
logic. Still $EO=0$. 
The CDAC state control is also triggered by $(P||N)=1$ (step 3 in Fig. \ref{fig_timing}), if $P=1$ then $D_{P0}=1$
and $D_{N0}=0$. If $N=1$ then $D_{N1}=1$ and $D_{P1}=0$. This
either adds, or subtracts charge from the top plate of the
CDAC. 


For the most significant bit the transition of $D_{P1},D_{P0},D_{N0},D_{N1}$ can be slow due to the
high capacitance, especially if the SAR runs at
a low voltage since the inverters will have lower driver strength. 
The advantage of this work is that the transition of $D_{P1}$ and $D_{N0}$ is included in the bit-cycling
clock generation loop. Since either $D_{P1}$ or $D_{N0}$ is guaranteed to transition
from high to low, these signals can be used to trigger comparator
reset.  When either $D_{P1}$ or $D_{N0} = 0$ then $M_{P6}$ or $M_{P7}$ turns on, and sets
node $B=1$, and consequently $CO = 1$ (step 4 in Fig. \ref{fig_timing}). The delay from $P||N=1$ to
$CO=1$ will depend on the process corner, CDAC capacitor variation, temperature,  and supply voltage. As a result the CDAC will be given sufficient
time to settle completely, independent of process, voltage, and
temperature (PVT). 


%-----------------------------------------------------------------
% Extend with simulation of the different times from cadence
%-----------------------------------------------------------------
Simulation of the voltage tolerance can be seen in Fig
\ref{fig_sim}. Fig \ref{fig_sim}(a) shows the core SAR ADC at a
$VDD=0.49V$. 
%At that voltage and room temperature the maximum measured sample rate
%is 2MS/s.
The first comparator decision is started by the sampling clock ($CK$),
seen at $1ns$ for low to high transition of $CK\_CMP$. At this low voltage
there is a long delay through the comparator to $P$ and $N$, and
further through the CDAC state control before $D_{N0}$ transitions to
zero. At that point $CO$ transitions high, and resets the
comparator. The fall-time from $90\%$ to $10\%$ of $D_{N0}$ is
$1.11ns$. In terms of time constants the fall-time is approximately
$ 2.2 \uptau$ for a single pole system. Thus the time constant for
node $D_{N0}$ is approximately $0.5ns$. The required settling
time for an accuracy within 1 LSB with 9-bit resolution is $-ln\frac{1}{512} \times tau <
6.23 \times 0.5ns < 3.12ns $. The simulated delay from $D_{N0}$ to $CK\_CMP$ is $7.11ns$, thus the CDAC has
sufficient settling time for 9-bit accuracy.

%-----------------------------------------------------------------
% High simulation
%-----------------------------------------------------------------
Fig \ref{fig_sim}b) shows the core SAR ADC at $VDD=0.71V$, in this
case the fall time of $D_{N0} = 0.315ns$, thus the required delay for
9-bit settling is
less than $6.23 \times \frac{0.315ns}{2.2} < 0.89ns$.  The
simulated delay from $D_{N0}$ to $CK\_CMP$ is
$1.68ns$ for Fig \ref{fig_sim}b) as a result there is sufficient
settling time for 9-bit accuracy.

\begin{figure}[tb]
\centerline{\includegraphics[width=\myfigwidth]{../graphics/fig_simulation.pdf}}
\caption{Simulation result of core SAR ADC at (a) 2MS/s and 0.49V and
  b) 20MS/s and 0.71V. }
\label{fig_sim}
\end{figure}


The maximum sample rate will depend on
PVT, thus in an industrial design one must ensure sufficient margin for
sample rate over PVT. The SAR architecture in this work is well suited to a
calibration loop where the supply voltage is adjusted until $EO$ of
the last logic stage is detected. In a Bluetooth\textsuperscript{\textregistered} low energy system this can be implemented at
radio power up. In this work, however, the voltage is fed from an external source, and the supply voltage calibration is a manual process. Another advantage of the clock generation loop in this work is that $CK\_CMP$ will
speed up for LSBs, since $D_{P1}$ and $D_{N0}$ will transition faster for
LSB stages, and the ripple path from $CI$ to $CO$ to
comparator is shorter.

\subsection{Comparator reset}
The comparator in Fig. \ref{fig_sar} is reset when $CK\_CMP=0$, which
occurs when $CO=1$, since $CK=0$, and
for the last stage $EO=0$ (steps 5 and 6 in Fig. \ref{fig_timing}). The comparator will set signals $P=N=0$, which turns on transistors $M_{P2}$ and $M_{P1}$, and sets $EO=1$ (step 7). This enables the next stage. It also locks the state of the
CDAC state control,
since $M_{N5}$ and $M_{N8}$ turn off. Also, $CO=0$ and in the end $CK\_CMP=1$ (step 8 and 9 in Fig. \ref{fig_timing}),
which clocks the comparator once more, and the next bit is decided.
The bit-cycling ends when $EO=1$ for the last stage.
