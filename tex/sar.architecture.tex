\subsection{Enable logic}
%-----------------------------------------------------------------
% Explain enable logic
%-----------------------------------------------------------------
The enable logic in Fig. \ref{fig_sar}(b) consists of two dynamic logic latches clocked by the $P$ and
$N$ outputs from the comparator. 
The enable logic is reset when $CK = 1$, at that point  $M_{P0}$ is on, while for the
first logic stage $M_{N0}$ is off, since $EI=\overline{CK}$,
accordingly node $A=1$ and $EO=0$, thus $EI=EO=0$ of all
subsequent stages. 

The first comparator decision is initiated by
$CK=0$,  at that point the first latch ($M_{N1},M_{N2},M_{N0},M_{P0}$)
in Fig. \ref{fig_sar}(b) is armed. When the
comparator makes a decision either $(P||N)=1$, which will turn on either
$M_{N2}$ or $M_{N1}$, thus $A=0$, which arms the
second latch ($M_{N3},M_{P1},M_{P2},M_{P3}$). When the comparator
resets reset $(P||N) = 0$, and $EO=1$, enabling the next stage.



\subsection{CDAC state control}
%-----------------------------------------------------------------
% Explain the CDAC control
%-----------------------------------------------------------------
The CDAC state control can be seen in Fig. \ref{fig_sar}(c).  The
sample clock also resets the state of the CDAC to $D_{P0}, D_{N1} = 1$ and
$D_{N0}, D_{P1} = 0$. When a stage is enabled $EI=1$ and then $(P||N)
= 1$  the state of the CDAC changes.  If $P=1$ then $D_{P0}=1$
and $D_{N0}=0$. If $N=1$ then $D_{N1}=1$ and $D_{P1}=0$. This
either adds, or subtracts charge from the top plate of the
CDAC. 

%-----------------------------------------------------------------
% Explain how the state is locked.
%-----------------------------------------------------------------
The CDAC state is locked when the next stage is enabled
$\overline{EO}=0$, this prevents subsequent transtions of $P$ and $N$ from
changing the state. $D_{P1}$ is used as the digital output of the stage,
and is captured by flip-flops outside the ADC core at $CK=1$.




\subsection{Clock generation loop}
%-----------------------------------------------------------------
% Introduce the clock generation loop
%-----------------------------------------------------------------
%One of the key contributions in this work is including the bottom
%plate of the CDAC in the bit-cycling clock generation
%loop. 
%-----------------------------------------------------------------
% Re explain why previous clock generation loops are problematic
%-----------------------------------------------------------------
Transition of $D_{P1}, D_{P0}, D_{N0}, D_{N1}$ of $LOGIC[8]$ can be slow due to the
high capacitance ($2^8 \cdot C_{UNIT}$), especially if the SAR runs at
a low voltage since the $VREF$ inverters will have low drive  strength. 
%-----------------------------------------------------------------
% Explain how we have a solution to the problem
%-----------------------------------------------------------------
The advantage of this work is that the transition of $D_{P1}$ and $D_{N0}$
is included in the bit-cycling
clock generation loop, as seen in
Fig. \ref{fig_sar}(d). Since either $D_{P1}$ or $D_{N0}$ is guaranteed to transition
from high to low, these signals can be used to trigger comparator
reset. 
When either $D_{P1}$ or $D_{N0} = 0$ then $M_{P6}$ or $M_{P7}$ turns on, and sets
node $B=1$, and consequently $CO=1$, as long as $EO=0$. The delay from $P||N=1$ to
$CO=1$ will depend on the process corner, CDAC capacitor variation, temperature,  and supply voltage. As a result the CDAC will be given sufficient
time to settle completely, independent of process, voltage, and
temperature (PVT). 

%-----------------------------------------------------------------
% Extend with simulation of the different times from cadence
%-----------------------------------------------------------------
Simulation of the voltage tolerance can be seen in Fig
\ref{fig_sim}. Fig \ref{fig_sim}(a) shows the core SAR ADC at a
$VDD=0.49V$. 
%At that voltage and room temperature the maximum measured sample rate
%is 2MS/s.
The first comparator decision is started by the sampling clock ($CK$),
seen at $1ns$ for low to high transition of $CK\_CMP$. At this low voltage
there is a long delay through the comparator to $P$ and $N$, and
further through the CDAC state control before $D_{N0}$ transitions to
zero. At that point $CO$ transitions high, and resets the
comparator. The fall-time from $90\%$ to $10\%$ of $D_{N0}$ is
$1.11ns$. In terms of time constants the fall-time is approximately
$ 2.2 \uptau$ for a single pole system. Thus the time constant for
node $D_{N0}$ is approximately $0.5ns$. The required settling
time for an accuracy within 1 LSB with 9-bit resolution is $-ln\frac{1}{512} \times tau <
6.23 \times 0.5ns < 3.12ns $. The simulated delay from $D_{N0}$ to $CK\_CMP$ is $7.11ns$, thus the CDAC has
sufficient settling time for 9-bit accuracy.

%-----------------------------------------------------------------
% High simulation
%-----------------------------------------------------------------
Fig \ref{fig_sim}b) shows the core SAR ADC at $VDD=0.71V$, in this
case the fall time of $D_{N0} = 0.315ns$, thus the required delay for
9-bit settling is
less than $6.23 \times \frac{0.315ns}{2.2} < 0.89ns$.  The
simulated delay from $D_{N0}$ to $CK\_CMP$ is
$1.68ns$ for Fig \ref{fig_sim}b) as a result there is sufficient
settling time for 9-bit accuracy.

%-----------------------------------------------------------------
% The fact that the process speeds up is also important
% - TBD: figure out where it fits
%-----------------------------------------------------------------
Another advantage of the clock generation loop in this work is that $CK\_CMP$ will
speed up for LSBs, since $D_{P1}$ and $D_{N0}$ will transition faster for
LSB stages, and the ripple path from $CI$ to $CO$ to
comparator is shorter. 
%-----------------------------------------------------------------
% Explain adoption to future industrial design
%-----------------------------------------------------------------
The maximum sample rate will depend on
PVT, thus in an industrial design one must ensure sufficient margin for
sample rate over PVT. The SAR architecture in this work is well suited to a
calibration loop where the supply voltage is adjusted until $EO$ of
the last logic stage is detected. In a Bluetooth\textsuperscript{\textregistered} low energy system this can be implemented at
radio power up. In this work, however, the supply voltage is fed from an
external source, and the supply voltage calibration is a manual
process.  

\begin{figure}[tb]
\centerline{\includegraphics[width=\myfigwidth]{../graphics/fig_simulation.pdf}}
\caption{Simulation result of core SAR ADC at (a) 2MS/s and 0.49V and
  b) 20MS/s and 0.71V. }
\label{fig_sim}
\end{figure}
