
% Version history:
% Version 1 (2016-11-22) = Rev 2859
% Version 2 (2017-01-24) = Rev 2888
% Version 3 (2017-03-02) = Rev 2893


\newcommand{\technm}{28 nm FDSOI}

\title{A Compiled 9-bit 20 MS/s 3.5 fJ/conv.step SAR ADC in {\technm} for
  Bluetooth Low Energy Receivers}
%
\author{Carsten~Wulff, \textit{Member, IEEE},
  Trond~Ytterdal, \textit{Senior Member, IEEE} }


\maketitle


\begin{abstract}
This paper presents a low power 9-bit compiled
successive-approximation register (SAR) analog-to-digital converter (ADC) for
Bluetooth low energy receivers. The
ADC is compiled from a SPICE netlist, a technology rule file, and an object definition file
into a design rule check (DRC) and layout versus schematic (LVS) clean layout and schematic in {\technm}. The compiled SAR ADC
reduces the design time necessary to port to a new
technology, and to demonstrate
technology porting the same SAR ADC architecture is compiled in {\technm}
with IO transistors. This work also includes a comparator
clock generation loop that \edit{uses} the bottom plate of the capacitive
digital-to-analog converter.  The proposed
compiled core transistor SAR ADC achieves state-of-the-art FoM of 2.7
fJ/conv.step at 2 MS/s, and 3.5 fJ/conv.step at
20 MS/s with an area of 0.00312 mm\textsuperscript{2}.

Carsten Wulff (Corresponding author), carsten.wulff@ntnu.no,
+4793200714, Trond Ytterdal, trond.ytterdal@ntnu.no,  Department of Electronics and Telecommunications, NTNU,
7491 Trondheim, Norway.
\end{abstract}
\begin{IEEEkeywords}
analog layout,  analog layout synthesis, ADC, analog-to-digital
conversion, FDSOI, successive approximation, low-power, Bluetooth low energy
\end{IEEEkeywords}

\section{Introduction} \label{introduction}

%-----------------------------------------------------------------
% Why are low power ADCs important in BLE recievers?
%-----------------------------------------------------------------
\IEEEPARstart{A}{} low-power, medium resolution analog-to-digital converter (ADC) is a key building block in
Bluetooth\textsuperscript{\textregistered} low energy \cite{ble} receivers. Low
power in the ADC is essential, as it can take up a large portion of the
receiver power budget, especially since two ADCs are usually required in receivers with quadrature down-conversion; one for
the real branch, and one for the imaginary branch.

%-----------------------------------------------------------------
% Why is reducing the design time important?
%-----------------------------------------------------------------
In a
Bluetooth\textsuperscript{\textregistered} low energy receiver the
design time of the ADC can be high compared to design time of the anti-alias filter,
low-noise amplifier, and mixer.
%-----------------------------------------------------------------
% What is state of the art within EDA tools?
%-----------------------------------------------------------------
Reducing design time of ADCs is an active research area, and part of
this research is analog layout generation.

%-----------------------------------------------------------------
% Why is analog layout generation so hard?
%-----------------------------------------------------------------
One of the challenges with analog layout generation is the large
solution space. A schematic designer has freedom
to choose from any architecture, any combination of transistors, and a
large number of transistor widths and lengths. The schematic
alone is not sufficient to fully determine the layout. For example, the
schematic does not contain placement information for layout, since the
optimum layout placement might be different from optimal schematic
placement.

Analog layout generation has a long history with works from the
previous century \cite{rijmenants89,bruce96}, but state-of-the-art
analog layout generation, as reviewed in \cite{martin13}, is not
widely adopted. More promising research avenues avoid the challenge of analog layout
generation from schematics, by not having a schematic.  Recently, ADCs have been compiled in a digital flow
\cite{weaver14,fahmy15}, and although the ENOB was less than 6-bit, it is an interesting approach. A similar
approach has been used successfully for all-digital PLLs \cite{deng15}.

%-----------------------------------------------------------------
% How does what we do make it easier to generate layout?
%-----------------------------------------------------------------
This work, first introduced at a conference \cite{wulff16}, presents a
method where the layout is not
generated based on drawn schematics. The ADC is described
using an approach borrowed from object oriented
programming.
A custom compiler is used to compile the ADC. The ADC is described in
a SPICE netlist, an
object definition file, and a technology rule file. The compiler outputs
a schematic and layout in SKILL\footnote{SKILL is the Lisp variant
  script language used in Cadence Virtuoso} and GDSII.
The output can be loaded into
Cadence Virtuoso and a standard design flow can be used for
parasitic extraction, simulation, and verification.

To reduce the solution space, the presented method uses a limited set of circuit blocks, a low
complexity ADC architecture, and proposes a technology
independent method for generating a design rule check (DRC) and layout
versus schematic (LVS) clean schematic and layout in multiple
technologies. Technology independence in this work has been
modeled with a tapeout of one ADC in {\technm} with
core transistors, and another in {\technm} with IO transistors.

%-----------------------------------------------------------------
% Why is it not enough to generate a ADC?
%-----------------------------------------------------------------
To demonstrate that an ADC can be compiled is not sufficient to
displace traditional analog design methodology. A compiled ADC is of
limited use unless it can be demonstrated that the performance can match
state-of-the-art ADCs.
%-----------------------------------------------------------------
% Explain how we plan to demonstrate technology independence?
%-----------------------------------------------------------------
To the best of
the authors knowledge, this work presents
the most efficient compiled ADC and achieves a FoM comparable to the
state-of-the-art \cite{harpe12,patil16,liu16}  (as surveyed in
\cite{murmann16}) with a smaller area, making it
perfectly suited for Bluetooth\textsuperscript{\textregistered} low
energy receivers.


%-----------------------------------------------------------------
% Explain organization of the paper
%-----------------------------------------------------------------
The remainder of this paper is organized as follows: Section \ref{architecture}  describes the
SAR ADC architecture, while Section \ref{compiler} describes the
compiler. \edit{ Section \ref{designmethod} describes the design methodology
using the compiler.} Measurement results are shown in Section \ref{measurements}
and conclusions given in Section \ref{conclusion}.




\section{SAR ADC Architecture} \label{architecture}
\moved{
%-----------------------------------------------------------------
% Introduce bit-cycling
%-----------------------------------------------------------------
The successive-approximation register (SAR) ADC architecture proposed
in this work includes a technique that \edit{uses} the bottom plate
of the capacitive digital-to-analog converter (CDAC) in the bit-cycling clock generation loop.
%-----------------------------------------------------------------
% Explain the problem with clock generation loops in prior art
%-----------------------------------------------------------------
Two clocks are commonly used in SAR ADCs. One to set the sample rate,
which in this work is applied externally to the ADC, and one bit-cycling
clock, which can be generated locally in the ADC. A typical SAR ADC
architecture can be seen in \reg{fig_prior}{a}, with a comparator,
SAR logic, and the CDAC. The comparator is
forced into reset during sampling ({$CK=1$}), and it triggers when the sample clock turns
off. It is common to use detection methods to determine when the
comparator has made a decision, and use this to trigger the subsequent resets of
the comparator. Such self-timed loops, shown in
\reg{fig_prior}{a},
are common in prior art \cite{harpe12,harpe12a,fredenburg12,kapusta13}.

One of the challenges in a self-timed loop is
to allow sufficient time for the CDAC to settle. Each time
the comparator makes a decision, one (or more) of the capacitors are
switched and the resulting transient must settle to the accuracy of the
ADC.  It is possible to use circuit
techniques like delay cells \cite{harpe12,harpe12a,fredenburg12}, or replica delays \cite{kapusta13} to adjust the time
between comparator decisions. Common to these techniques is that they try to model the
required CDAC settling time, and allow enough time before
the comparator makes the next decision.
%-----------------------------------------------------------------
% Explain how what we do is the solution to clock loops
%-----------------------------------------------------------------
In this work, we use the bottom plate of the
CDAC directly in the comparator clock generation loop to improve robustness,
and ensure sufficient time for CDAC settling, as shown in \reg{fig_prior}{b}.
}
\begin{figure}[t]
\centering
\includegraphics[width=\myfigwidthc]{fig_prior_art}
\caption{Comparator clock generation in: (a) Prior works, (b) This work.}
\label{fig_prior}
\end{figure}

%-----------------------------------------------------------------
% Introduce the sar architecture
%-----------------------------------------------------------------
The SAR ADC architecture used to demonstrate the compiler is shown in
\reg{fig_sar}{a}.  It has NMOS
boot-strapped input switches, a strong-arm comparator with kick-back compensation
\cite{varzaghani13}, and a metal-oxide-metal finger CDAC.

\begin{figure*}[t]
\centerline{\includegraphics[width=\myfigwidthl]{fig_sar_logic}}
\caption{Proposed SAR ADC architecture with: (a) Block diagram, (b) Enable
  logic, (c) CDAC state control, (d) Clock generation.}
\label{fig_sar}
\end{figure*}


%-----------------------------------------------------------------
% Explain timing in the SAR
%-----------------------------------------------------------------
Sampling of the differential input signal ($V_P - V_N$) is
controlled by the sample clock $CK$. The sample clock has a
duty-cycle of less than 25 \% to increase the time available for the
SAR algorithm. The first SAR logic block
($LOGIC[8]$) is enabled when $CK=0$, the next logic block ($LOGIC[7]$) is
enabled when $LOGIC[8]$ has completed and sets $EO=1$. The bit-cycling continues until $EO=1$ for the last logic block
($LOGIC[0]$).

%-----------------------------------------------------------------
% Explain the comparator
%-----------------------------------------------------------------
\deleted{
The comparator clock ($CK\_CMP$) is generated by the loop that
starts with the comparator outputs ($P$ and $N$) into the SAR logic
blocks and out from the $CO$ output. A pulse
signal is generated internally in the SAR logic, and the $CI$ and $CO$
form a chain of $OR$ gates.
% with the last connected
%through NOT gate $X1$ and AND gate $X2$ to $CK\_CMP$.  The comparator
%outputs $P$ and $N$ are reset to zero when $CK\_CMP$ is zero.
%When the comparator makes a decision, either $P$ or $N$
%will be high.
The digital output from each stage is $D_{P1}$.
}

%-----------------------------------------------------------------
% Explain the CDAC
%-----------------------------------------------------------------
The bottom plate of the CDAC capacitors
are controlled directly by the $D_{P0}$, $D_{P1}$, $D_{N0}$ and
$D_{N1}$ signals, which switch between ground and the reference
voltage. The reference voltage is at the same voltage as the supply
voltage. The
first five stages of the ADC use split monotonic switching \cite{liu10a} to reduce the common
mode variation, while the last four stages use monotonic switching
\cite{liu10}.
\edit{ In a monotonic based scheme, an 8-bit CDAC is
sufficient. The SAR ADC in this
work is intended as building block for noise-shaped SAR ADCs
\cite{fredenburg12}, where the last residue is needed, thus it uses
a 9-bit CDAC.}

%-----------------------------------------------------------------
% Explain the unit transistor, and the compromise
%-----------------------------------------------------------------
\secedit{
The compiled ADC uses unit transistors for all blocks in the design, as explained below.
}
\edit{
One unit transistor size is used for both PMOS and NMOS.
The \textit{W = 258 nm} and \textit{L =  30 nm} for the core
transistors, and \textit{W =
344 nm}  and  \textit{L = 180 nm} for the IO transistors.  A small unit
transistor was chosen to reduce the power consumption in the
comparator and digital logic. The schematic and
sizing of the comparator \cite{varzaghani13} is shown in
\req{fig_comp}. The boot-strapped switch transistor has four unit transistors in parallel.
}

\begin{figure}[tb]
\centerline{\includegraphics[width=\myfigwidth]{fig_comparator}}
\caption{\edit{Strong-arm comparator with kick-back
  compensation. Transistors without numbers are unit size, while
  transistors with numbers are parallel combinations of unit transistors.}}
\label{fig_comp}
\end{figure}


%-----------------------------------------------------------------
% Introduce the SAR logic
%-----------------------------------------------------------------
The SAR logic  consists of three parts:
enable logic (\reg{fig_sar}{b}), CDAC state control
(\reg{fig_sar}{c}), and clock generation loop
(\reg{fig_sar}{d}).
%\subsection{SAR Logic Reset}
During sampling of the ADC input the SAR logic is reset ($CK=1$). In the
enable logic of the first  stage $EI=\overline{CK}=0$, and node $A=1$,
while $EO=0$. Thus $EI=EO=0$ of all subsequent stages. The CDAC
state control has $D_{P0}=D_{N1}=0$ while $D_{N0}=D_{P1}=1$. In the clock generation
loop of the first stage $CI=0$, and since node $B=0$, then
$CO=0$. Accordingly $CI=CO=0$ for all subsequent stages.

%\subsection{Comparator Decision}
The first comparator decision is initiated by $CK=0$ (steps 1 and 2 in \req{fig_timing}). At that point the
first latch ($M_{N0}-M_{N2}, M_{P0}$) in the enable logic is armed, and as soon as $(P||N)=1$,
then $A=0$. This arms the second latch ($M_{N3},M_{P1}-M_{P3}$) in the enable
logic. Still $EO=0$.
\deleted{
The CDAC state control is also triggered by $(P||N)=1$ (step 3 in
\req{fig_timing}).
%If $P=1$ then $D_{P0}=1$
%and $D_{N0}=0$. If $N=1$ then $D_{N1}=1$ and $D_{P1}=0$.
%This
%either adds, or subtracts charge from the top plate of the CDAC.
}

\begin{figure}[t]
\centerline{\includegraphics[width=\myfigwidth]{fig_timing}}
\caption{Timing diagram of the SAR logic.}
\label{fig_timing}
\end{figure}

\deleted{
%The transition of $D_{P1},D_{P0},D_{N0},D_{N1}$ can be slow for the
%most significant bit due to the
%high capacitance, especially if the ADC runs at
%a low voltage since the inverters (\reg{fig_sar}{c}) will have lower
%drive strength.
The advantage of the bit-cycling clock generation loop in this work, is
that it includes the transition of $D_{P1}$ and $D_{N0}$.
}
\deleted{
Since either $D_{P1}$ or $D_{N0}$ is guaranteed to transition
from high to low, these signals can be used to trigger comparator
reset.  When either $D_{P1}$ or $D_{N0} = 0$ then $M_{P6}$ or $M_{P7}$ turns on, and sets
node $B=1$, and consequently $CO = 1$ (step 4 in \req{fig_timing}).

%The delay from $P||N=1$ to
%$CO=1$ will depend on the process corner, CDAC capacitor variation, temperature,  and supply voltage. As a re%sult the CDAC will be given sufficient
%time to settle, independent of process, voltage, and
%temperature (PVT).
}

%\subsection{Comparator Reset}
The comparator in \req{fig_comp} is reset when $CK\_CMP=0$, which
occurs when $CO=1$, since $CK=0$, and
for the last stage $EO=0$ (steps 5 and 6 in
\req{fig_timing}). The comparator will set signals $P=N=0$, which
turns on transistors $M_{P2}$ and $M_{P1}$, and sets $EO=1$ (step
7). This enables the next stage, and locks the state of the
CDAC state control,
since $M_{N5}$ and $M_{N8}$ turn off. Also, $CO=0$ and in the end $CK\_CMP=1$ (step 8 and 9 in \req{fig_timing}),
which clocks the comparator once more, and the next bit is decided.
\deleted{
The bit-cycling ends when $EO=1$ for the last stage.
}


%\subsection{Simulation of clock generation loop}
%-----------------------------------------------------------------
% Extend with simulation of the different times from cadence
%-----------------------------------------------------------------
% \reg{fig_sim}{a} shows a simulation of the core SAR ADC at a supply
% voltage of 0.49 V.
%At this low voltage
%there is a long delay through the comparator to $P$ and $N$, and
%further through the CDAC state control before $D_{N0}$ transitions to
%zero.  The fall-time from 90 \% to 10 \% of $D_{N0}$ is
%1.11 ns. In terms of time constants the fall-time is
%$ 2.2 \uptau$ for a single pole system.\footnote{ Calculated from $t =
%  \uptau[ln(\frac{1}{1-0.9}) - ln(\frac{1}{1-0.1})] = 2.2\uptau$} Thus the time constant for
%node $D_{N0}$ is approximately 0.5ns. The required settling
%time for an accuracy within 1 LSB with 9-bit resolution is $-ln\frac{1}{512} \times \uptau <$
%6.23 $\times$ 0.5 ns $=$ 3.12 ns. The simulated delay from $D_{N0}$ to
%rising edge of $CK\_CMP$ is 7.11 ns, thus the CDAC has
%sufficient settling time for 9-bit accuracy.

%-----------------------------------------------------------------
% High simulation
%-----------------------------------------------------------------
%\reg{fig_sim}{b} shows the core SAR ADC at a supply voltage of 0.71 V. In this
%case the fall time of $D_{N0}$ is 0.315 ns, thus the required delay for
%9-bit settling is
%less than $6.23 \times \frac{0.315ns}{2.2} =$ 0.89 ns.  The
%simulated delay from $D_{N0}$ to rising edge of $CK\_CMP$ is
%1.68 ns for \reg{fig_sim}{b}, as a result there is sufficient
%settling time.

%\begin{figure}[t]
%\centerline{\includegraphics[width=\myfigwidth]{fig_simulation}}
%\caption{Simulation result of core SAR ADC at: (a) 2 MS/s and 0.49 V,
%  (b) 20 MS/s and 0.71 V.}
%\label{fig_sim}
%\end{figure}


%The maximum sample rate of the ADC will depend on
%PVT, thus in an industrial design one must ensure sufficient margin for
%sample rate over PVT. The SAR ADC architecture in this work is well suited to a
%calibration loop where the supply voltage is adjusted until $EO$ of
%the last logic stage is detected. In a Bluetooth\textsuperscript{\textregistered} low energy system this can be implemented at
%radio power up. In this work, however, the supply voltage is fed from
%an external source, and the supply voltage calibration is a manual
%process.



\section{Compiler} \label{compiler}
%-----------------------------------------------------------------
% Reiterate what the problem is
%-----------------------------------------------------------------
The key contribution in this work is to speed up the design time
of SAR ADCs, and reduce the effort necessary to port a design to
another technology. Schematic capture, simulation,  layout, and
parasitic netlist simulation, is a time consuming endeavor for
ADCs. For
charge based SAR ADCs the small unit capacitance make them sensitive to poor
 layout. It is common to spend time tracking down fF
parasitics, and redoing  layout. The design time would be
shorter if one could generate DRC/LVS clean layout in minutes, instead
of days. Rapid ADC generation would allow for more iterations before tapeout, and
could even allow automatic exploration of the design space. It could
also enable rapid porting to another technology.

%-----------------------------------------------------------------
% Re-explain how this work fixes the problem
%-----------------------------------------------------------------
The compiler in this work (a Perl script) compiles a core transistor
SAR ADC into GDSII in $2.7$ seconds on a Macbook Air with 1.7GHz Intel Core i7. The Perl version of the
compiler is closed source, but there is a reduced feature set open
source version written in C++ \cite{ciccreator16}, which
compiles a SPICE netlist, an object definition file, and a rule file of a core transistor SAR ADC into GDSII in $0.35$ seconds.

%-----------------------------------------------------------------
% Introduce the SPICE netlist
%-----------------------------------------------------------------
\deleted{
The SPICE netlist does not
contain transistor lengths or transistor widths, rather it contains
references to unit transistors.  The unit transistor can be
series connected, or parallel connected.
}

%-----------------------------------------------------------------
% Introduce the object definition file
%-----------------------------------------------------------------
The object definition
file is written in JavaScript Object Notation (JSON) \cite{json16}, a common text
data format, which is supported in most programming languages. The object definition file, which contains the definition of the unit transistor, does not contain absolute
dimensions, thus it is technology independent.  The object definition file also defines
the routing of blocks in the ADCs.
%-----------------------------------------------------------------
% Introduce the rule file
%-----------------------------------------------------------------
Absolute dimensions, and technology specific information are defined
in a rule file.


The remainder of this Section is organized as follows, the rule file is described in Section \ref{rule}, while the unit
transistor is described in Section \ref{unittransistor}, and the unit
capacitor in Section \ref{unitcapacitor}. The place and route of the
design is described in Section \ref{placeroute}, while the top level
SAR ADC layout is described in Section \ref{overallayout}.

\subsection{Technology rule file} \label{rule}
%-----------------------------------------------------------------
% Introduce the rule file
%-----------------------------------------------------------------
The rule file contains the GDSII layer numbers, GDSII data-type, layer
material definitions, what cuts \edit{(vias)} to use for transitions
between layers, and design rules.

% -----------------------------------------------------------------
% Lambda rules
%-----------------------------------------------------------------
All rules, like poly width, cut size, metal width, metal
spacing and metal cut enclosure are based on a dimensionless parameter
$\mathit{GAMMA}$. The layout rules in this work are inspired by early work on dimensionless layout
rules, or LAMBDA rules, where the LAMBDA parameter was equal to half the minimum feature
size ($\mathit{F/2}$) \cite{mead.conway}. The dimensionless rule approach was further extended
to nano-scale technologies with $\mathit{GAMMA = F/4}$ in
\cite{lewyn09,lewyn11}.

In this
work the $\mathit{GAMMA}$ has been redefined to $\mathit{F/6}$ to allow better
granularity of the layout rules.
%-----------------------------------------------------------------
% Explain the rules
%-----------------------------------------------------------------

In the rule file for  {\technm}
 there are 140 unique rules, but 95 of those are for the
different metals and cuts (spacing, width, enclosure).

%-----------------------------------------------------------------
% Explain materials and stack
%-----------------------------------------------------------------
The compiler can automatically insert cuts between layers, and to build
the cuts the material type (active, cut, metal, poly), and the connect stack (PO to CO to
M1, M1 to VIA1 to M2), is defined in the
rule file for each of the layers.


\subsection{Unit transistor} \label{unittransistor}
%-----------------------------------------------------------------
% Explain that the objdef uses classes
%-----------------------------------------------------------------

The structure of the object definition file is
inspired by object oriented programming, where all objects are
instances of a class. The class is defined in the compiler,
and the compiler has classes for transistors, capacitors, resistors,
digital cells and complex cells (CDAC, SAR ADC).

%-----------------------------------------------------------------
% Explain why traditional pcells are bad
%-----------------------------------------------------------------
The layout rules for nano-scale
transistors contain complex spacing rules. In the {\technm}
design rule deck there are close to 5000 design rules.
Not all these rules relate to transistors, but it is complex to design a DRC clean
programmable transistor cell that can support multiple technologies,
from multiple foundries.
%-----------------------------------------------------------------
% Explain our solution, the unit transistor
%-----------------------------------------------------------------
Instead of a complex programmable transistor cell this work
implements a simplified transistor cell.  The layout of the transistor can be
seen in \req{fig_dmos}.

\begin{figure}[t]
\centering
\centerline{\includegraphics[width=\myfigwidtha]{fig_dmos}}
\caption{Unit transistor layout.}
\label{fig_dmos}
\end{figure}

%-----------------------------------------------------------------
% Introduce grids
%-----------------------------------------------------------------
To avoid complex rules the rule file defines two numbers, the vertical grid size and
the horizontal grid size. \edit{For the ADCs in {\technm} the grids are the same (\textit{86
    nm}), but two numbers are defined in the compiler for future flexibility. The
compiler also supports modification of the grid on a per cell basis,
and multiple unit transistors in a design.}

%-----------------------------------------------------------------
% Explain how it solves the complexity problem
%-----------------------------------------------------------------
The grid size controls the spacing
between poly gates, the active enclosure of cuts, the metal enclosure of
cuts, the dummy poly to active spacing and the poly to cut
spacing. To improve
manufacturability the transistor has two cuts for drain, source and poly,
regular poly pitch to improve poly critical dimension (CD) and
enlarged metal one rectangles to satisfy minimum metal
area DRC rules.
There are vertical routing channels between
active and gate, and gate and bulk contact to simplify
routing. On neighboring transistors the dummy poly can be overlapped to
ensure regular poly pitch over large distances.

%-----------------------------------------------------------------
% Introduce how transistors are defined
%-----------------------------------------------------------------
The unit transistor is described in the object definition file as an ASCII code as shown in
\req{fig_dmos_json}.
\begin{figure}[t]
\centering
\centerline{\includegraphics[width=2.5in]{fig_dmos_json}}
\caption{\edit{Object definition of the unit transistor (DMOS) and the PMOS (PCHDL).}}
\label{fig_dmos_json}
\end{figure}
%-----------------------------------------------------------------
% Explain how the ASCII works
%-----------------------------------------------------------------
The ASCII has a $ N x M $ matrix where each cell
is one grid in size. A $-$ character means empty space, $x$
means the grid is filled. $m$ means the grid is filled horizontally,
but the height of the rectangle is the minimum routing width for that
layer. For $w$ the grid is also filled horizontally, but the height
of the rectangle is the normal routing width for that layer. $K$, $C$,
and $c$ make cuts to the next layer. $D,G,S,B$ are the
familiar ports of the transistor.

%-----------------------------------------------------------------
% Introduce how PMOS and NMOS differs very slightly, same size
%-----------------------------------------------------------------
The PMOS and NMOS have the same physical dimensions, thus they
differ only by their respective implant layers and well definitions. The
compiler supports the object oriented programming concept of
inheritance, where an object can inherit all features of a
parent. One example is the PMOS in \req{fig_dmos_json}, which inherits the
DMOS. The function \textit{addEnclosures} searches the parent
cell and finds the OD rectangles, adds P-implant layer, and low threshold
voltage layer with enclosures defined in the rule file.  This object
oriented design approach allows complex cells to be constructed
from small modifications of existing cells.

To make the unit transistor DRC clean requires an
iterative process where the transistor layout is compiled, DRC checks
are run, and grid sizes adjusted until the transistor is DRC
clean. With this simplified transistor there is a risk that it will
not fit all technologies, but so far DRC clean transistors have been
compiled in {\technm}, 28 nm, 65 nm and 55 nm.



\subsection{Unit capacitor} \label{unitcapacitor}
%CAPs
% 1
% 1.978
% 4.263
% 7.836
% 7.83
% 8.201
% 16.04
% 16.13
% 32.16
% 32.12
% 64.29
% 64.25
% 128.6
% 128.7
%-----------------------------------------------------------------
% Introduce the unit capacitor
%-----------------------------------------------------------------
\edit{The building block for the 9-bit CDAC is a 5-bit CDAC cell, with an additional
unit capacitor to make 32 unit capacitors. A 3D view \cite{GDS3D} can
be seen in
\reg{fig_cap32}{a}, and the cell in \reg{fig_cap32}{b}. } The bottom
plate is a metal finger in metal four and metal
three. The capacitor top plate surrounds the bottom
plate fingers in metal four.  The bottom plate routing, in metal one,
is covered by a ground shield in metal two to reduce the parasitic
coupling to the top plate. \edit{
The compiler has a built-in class to make this cell, and the class gets the metal width, capacitor
finger spacing and metal route spacing from the rule file.

The four MSB in the 9-bit CDAC uses 15 cells (bit 8 = 8 cells, bit 7 = 4 cells, bit 6 =
2 cells, bit 5  = 1 cell). Bit 4 uses one cell, but half the capacitors
are connected to ground. Bit 3, bit 2, bit 1 and bit 0 share a cell,
thus in total 17 cells are used in the 9-bit CDAC. The netlist for the IO ADC with the
detailed CDAC cell arrangement is available at \cite{ciccreator16}.

The total capacitance seen from the top plate node of the 9-bit CDAC is
simulated at 169 fF. The parasitic capacitance to ground is simulated at
67 fF. The remaining 102 fF is the combination of the 512 unit
capacitors. The unit capacitor is simulated at 0.2 fF in {\technm}.
}


\begin{figure}[t]
\centerline{\includegraphics[width=\myfigwidthb]{fig_capacitors_vertical}}
\caption{\edit{Layout of the 5-bit CDAC cell used in the 9-bit CDAC with: (a) A
  3D view, (b) The 32 unit capacitor cells.}}
\label{fig_cap32}
\end{figure}


\subsection{Place and Route} \label{placeroute}
%-----------------------------------------------------------------
% Introduce the netlist
%-----------------------------------------------------------------
The layout of the enable logic from \reg{fig_sar}{b} can be seen in
\req{fig_sarmx},  with a 3D view \cite{GDS3D} in
\reg{fig_sarmx}{a}. The enable logic layout in \reg{fig_sarmx}{d} has been compiled from the  input files into
{\LaTeX} \cite{latex16} with \textit{cic2tikz} \cite{ciccreator16}  to improve figure
readability, thus it does not include the implant layers, or well definitions.
\begin{figure*}[t]
%\centering
  \centerline{\includegraphics[width=\myfigwidthl]{fig_saremx}}
\caption[Caption without FN]{The enable logic with: (a)  3D layout, (b) SPICE netlist (c) Object definition (d) Layout.}
\label{fig_sarmx}
\end{figure*}

%-----------------------------------------------------------------
% Explain key features
%-----------------------------------------------------------------
The placement of instances in the
layout is determined by the order of instances in the SPICE netlist
(\reg{fig_sarmx}{b}). The
instance name (i.e. $MN0$) is used to
group elements. It is the group name, defined as the characters up to
the first number (i.e $MN$), that determines whether row or column number
is incremented.  The first transistor, $MN0$, is placed in
column zero and row zero, as seen in \reg{fig_sarmx}{d}. If the next instance has the same group name the
row number is incremented. Thus all $MN$ transistors will be placed on
top of each other. If the
next instance has a different group name, for example $MP$, the row
number is reset and the column number is incremented. For example, instance
$MP0$ will be placed in row zero and column one, while $MP1$ to $MP3$
 are placed on top of $MP0$.

%-----------------------------------------------------------------
% Introduce routing
%-----------------------------------------------------------------
The compiler does not contain an auto-router. It contains
routing instructions to route circuits, thus a designer has to specify exactly the
necessary routes for the circuit blocks. The compiler does, however, contain
instructions to find ports, perform simple routes, and do layer stack
transitions.

The routes are defined in the object definition file, and the object
definition of the enable logic is shown in
\reg{fig_sarmx}{c}. The enable logic is an object of a class that
has support functions for digital cells. For example, the class automatically adds routes in metal
four for VDD and VSS, as seen in \reg{fig_sarmx}{a}.
Only the
outline is shown for the VDD and VSS in \reg{fig_sarmx}{d} to make the metal one routing visible.

There are two types of route instructions in \reg{fig_sarmx}{c},
a connectivity route, and  a directed
route. A connectivity route can search the circuit for instance ports, and
find all instance ports that belong to a net name. The net name
match is performed with regular expressions
\cite{masterRegex}. Route number 3 in
\reg{fig_sarmx}{c} is a connectivity route
for net
$EO$ between $M_{N3}$ and $M_{P1}$. The instructions for this route are {\small\texttt{['M1','EO','--|-','onTopR']}}, which
translates to: Find all instance ports in metal one ({\small\texttt{'M1'}}) that match the regular
expression {\small\texttt{'EO'}}, use rightmost port as the start of the route ({\small\texttt{'onTopR'}}),  and route
in a left, up or down, left pattern ({\small\texttt{'--|-'}}) to the other ports. This
forms the route marked with $3$ in \reg{fig_sarmx}{d}.

For the net $A$ in \reg{fig_sarmx}{b} the connectivity route is
not sufficient, as a connectivity route
would route from the gate of $MP3$ to the gate of $MN3$ in metal one,
thus creating a short. The directed route is used in these cases to
customize the route. For net $A$ there are three directed routes
(route number 6, 7, 8). The instructions for route number 6 is
{\small\texttt{['PO','A','MN3:G-MP3:G']}}, which translates to: Find
instance  {\small\texttt{'MN3'}}, check if the instance has a port
called  {\small\texttt{'G'}} in poly, and route
in poly ({\small\texttt{'PO'}}) horizontally to an instance with name  {\small\texttt{'MP3'}} and port  {\small\texttt{'G'}}.

%-----------------------------------------------------------------
% Justify the time consumption of the SAR routing
%-----------------------------------------------------------------
The complete object definition file contains object definitions for all the blocks in
the ADC. Most of the lines in the object definition file are route
instructions, and it requires effort from an analog designer to
define the routes of a complete ADC, but this is a one time
effort. Once the routes are defined they can scale to multiple
technologies without change. The reason it scales is because route information is void
of technology information.



\subsection{Top level SAR ADC layout} \label{overallayout}
The object definition of the SAR ADC top level \edit{uses} a SAR ADC class in
the compiler. The layout placement in this class is controlled
by the SPICE netlist, but the placement algorithm is different from
the digital cells. The SAR ADC class assumes that any
instance in the SPICE netlist with a group name $XCDAC$ is a CDAC. Any instance with group name $XB$ is assumed to
be a input switch, and instances with group name $XA$ are digital
cells. Based on these assumptions the compiler places the input switches at the
bottom, CDAC in the middle, digital cells at the top.
%-----------------------------------------------------------------
% Bridge to measurements
%-----------------------------------------------------------------
A compiled layout of an IO transistor SAR ADC can be seen in
\reg{fig_toplevel}{a}, and a core transistor SAR ADC in
\reg{fig_toplevel}{b}. The layout in \req{fig_toplevel} was compiled from the
input files into Encapsulated Postscript (EPS)\cite{eps92} with \textit{cic2eps}
\cite{ciccreator16} to improve figure readability, thus it does not
include cuts, implant layers, or well definitions.

To port the SAR ADC from core transistors to IO transistors
required some changes to the compiler inputs. The
metal spacing was changed in the rule file, which is why the CDAC in
the IO ADC is taller than in the core ADC. The unit transistor was redefined,  and PMOS and NMOS
\edit{were} modified to change the implant layers. There were no changes to
the SPICE netlist. Less than 5 \% of the lines in the input files required
changes to compile the IO transistor ADC from the core transistor ADC
input files.



\begin{figure}[t]
\centerline{\includegraphics[width=\myfigwidth]{fig_toplevel}}
\caption[Caption without FN]{\edit{Top level layout for: (a) An IO SAR
  ADC, (b) A Core SAR ADC.}}
\label{fig_toplevel}
\end{figure}



%\subsection{Porting to another technology} \label{port}

\edit{
\section{Design Methodology} \label{designmethod}

The ADC design in this work started with an architecture exploration
using
\secedit{
hand analysis and schematics, as shown in
\req{fig_cic}.
Compiled cells (5-bit CDAC, transistors) were used as
schematic building blocks for more complex cells (9-bit CDAC, comparator). }
When a suitable architecture
was found, the initial netlist was \secedit{modified to control
  placement of instances in the layout}, and the routing instructions implemented. As the
design progressed, more and more of the ADC was compiled. In the end,
the ADC was fully described by the SPICE netlist, object
definition file and technology file.
The architecture schematic was no longer needed, although it was kept
for most circuits to visualize the implemented architecture.
Testbenches and extracted parasitic netlist simulations were used to
verify the physical implementation. It was assumed that the physical verification was sufficient to verify the compiler code.
\moved{
The complete object definition file that
was used to generate the core transistor ADC in this work consists of
1266 lines of code, but that file generates 21 different SAR ADC variants,
with 8-bit, 9-bit, 10-bit and 11-bit CDACs.}
The design effort for
the initial compiled SAR ADC is estimated at 400
hours, including development of the compiler code.

It is reasonable to expect that some redesign must be done after
the initial compile, or indeed porting to another technology.
The advantage of this work for design and redesign, is
that changes to the circuit can be done in minutes, and recompiled in seconds.
For example, changing the width of the unit transistor, as shown in
\reg{fig_wideadcs}{b}. This allows the analog designer to start
simulation on extracted netlist early in the design flow, and primarily
do the optimization for performance, process,
voltage and temperature on extracted netlist.

The analog designer can also quickly adopt to new design
requirements. A  noise-shaped SAR ADC (11-bit ENOB simulated) \cite{garvik15}
was designed with a variant of the 9-bit core ADC in this work as a building
block. In a noise-shaped SAR ADC the thermal
noise sampled on the CDAC is
not shaped by the loop filter, thus \cite{garvik15} required more
CDAC capacitance. An example of a 9-bit SAR ADC
variant with more CDAC capacitance, is shown in \reg{fig_wideadcs}{c}.

In the authors experience, there is sufficient support in the compiler classes
to place and route most analog circuits, but the compiler can only use
cells that it generates. It cannot use standard-logic cells, or
layout cells drawn with traditional analog design methodology.

The compiler was also used for support circuits
outside the ADCs. A digital input pad circuit, shown in \req{fig_methodology}, uses the standard compiler
classes. Some analog circuits may
require custom compiler classes, and a
combination of analog design expertise and programming expertise is
needed to code the classes.

In \cite{garvik15} compiled cells were used as building blocks,
\secedit{
similar to the architecture exploration in this work,
but the top level layout }was drawn in Cadence Virtuoso. The advantage of the compiler for
complex analog circuits, for example Bluetooth\textsuperscript{\textregistered} low
energy receivers, is that key sub-blocks can be quickly
compiled. As a result, the analog designer can focus the design effort on
the difficult parts.
}


\begin{figure}[tb]
\centerline{\includegraphics[width=\myfigwidthb]{fig_process}}
\caption{\secedit{Design methodology}}
\label{fig_cic}
\end{figure}


\begin{figure}[tb]
\centerline{\includegraphics[width=\myfigwidthb]{fig_wideadcs}}
\caption{\edit{Layout variants of the core ADC with: (a) The original core
    ADC, (b) A core ADC with wide
    transistors ($1.72 \mu m$), (c) A core ADC with more CDAC capacitance.}}
\label{fig_wideadcs}
\end{figure}



\begin{figure}[tb]
%\centerline{\includegraphics[width=\myfigwidth]{fig_methodology}}
\centerline{\includegraphics[width=\linewidth]{fig_methodology}}
\caption{\edit{Digital input pad circuit with:
    (a) Schematic, (b) SPICE netlist, (c) Layout, (d) Object definition.}}
\label{fig_methodology}
\end{figure}

\section{Measurement Results} \label{measurements}
The micrograph can be seen in \reg{fig_die}{a}, and the die measures
1.04 mm $\times$ 1.04 mm without seal-ring.  The layout is shown in
\reg{fig_die}{b}, with the IO transistor ADC and core transistor ADC locations
indicated by arrows. On this die there are seven other compiled SAR ADCs with
8-bit, 10-bit and 11-bit resolutions.
The prototype is fabricated in a {\technm} process with an area of
39 $\mu$m  x 80 $\mu$m = 0.00312 mm\textsuperscript{2} for the core
transistor SAR ADC, and 40 $\mu$m x 106 $\mu$m =
0.00424 mm\textsuperscript{2} for the IO transistor SAR ADC, including logic, comparator, CDAC,
and input switch.
\begin{figure}[t]
\centerline{\includegraphics[width=\myfigwidtha]{fig_diephoto}}
\caption[Caption without FN]{Overall layout with: (a) Micrograph, (b) Chip Layout.}
\label{fig_die}
\end{figure}

%-----------------------------------------------------------------
% Explain measurements of core SAR
%-----------------------------------------------------------------
The input signal to the ADCs was supplied from a R\&S SML
signal generator through a 5th-order passive band-pass
filter, and the input frequency was selected for coherent sampling. A balun
was used to convert from single ended to differential, and the balun common
mode voltage was supplied externally through a resistive divider. The sample clock was
supplied from a second R\&S SML signal generator at twice the ADC
sample rate, and an on-chip sine-to-square circuit with a pulse-picking
divide-by-two  was used to generate the 25 \% duty-cycle. The
parallel output data was captured with an RTE1022 oscilloscope with
mixed-signal option. The data was post processed in MATLAB, and a
Hanning window was used before the FFT.

\begin{figure*}[t!]
  \centerline{\includegraphics[width=\myfigwidthl]{fig_core_measurements}}
  \caption{Measurements for core transistor SAR ADC with: (a) Spectrum
    at 20 MS/s, (b) Spectrum at 2 MS/s, (c) Peak ENOB as a function of
    supply voltage, (d) SNDR and SFDR as a function of input frequency
    \edit{at 20 MS/s}.}
  \label{fig_core}
\end{figure*}
Three chips have been measured for the core transistor SAR ADC with
similar performance. The core
transistor SAR ADC in this work achieves
a FoM\footnote{$FoM = P/(2^{ENOB} \times fs)$} of 3.5 fJ/conv.step
at 20 MS/s, shown in \reg{fig_core}{a}, and 2.7 fJ/conv.step at 2 MS/s (\reg{fig_core}{b}) placing it among the most power
efficient and area efficient ADCs for its sample rate and
resolution.

\edit{ The ENOB of the core ADC is limited by thermal noise. The comparator is
  the likely source of the thermal noise, as the ENOB can be changed
  by varying the input common mode voltage.}
The core transistor SAR ADC is functional from
80 kS/s - 80 MS/s and operates from a single  0.4 V - 1.1 V supply voltage, as
shown in \reg{fig_core}{c}. The SNDR and SFDR as a function of frequency can
be seen in \reg{fig_core}{d}.
\secedit{
The measured power consumption of the core ADC is
15.87 $\mu$W at 20 MS/s and 0.94 $\mu$W at 2 MS/s.
}
\edit{
  The measured power consumption of the core ADC CDAC is approximately
  44 \% at 20 MS/s and approximately 47 \% at 2 MS/s.
  The simulated power consumption of the core ADC CDAC is 42 \% at 20 MS/s and
  44 \% at 2 MS/s. The simulated power consumption of the comparator is
  25 \% at 20 MS/s and 23 \% at 2 MS/s.
}


%-----------------------------------------------------------------
% Explain measurements of IO SAR
%-----------------------------------------------------------------
The spectrum of the IO transistor SAR ADC is shown in
\req{fig_io}.\footnote{Due to limitations for the IO SAR ADC it is not possible to separate
the current in pad drivers, and ADC current. The
load on each pin is at least 4pF, as specified on the RTE1022
oscilloscope. Accordingly the current for the IO SAR ADC is estimated at
ADC current  =  542 $\mu$A - average(bit transitions x 10 MHz x 4 pF x
1.6 V).}
%-----------------------------------------------------------------
% Introduce comparison to prior art
%-----------------------------------------------------------------
\begin{figure}[t]
\centerline{\includegraphics[width=\myfigwidth]{fig_io10msps}}
\caption{Spectrum of IO transistor SAR ADC at 10 MS/s.}
\label{fig_io}
\end{figure}
Table \ref{tab_prior} shows a summary and
comparison of key measurements with prior-art.

\begin{table}[tb]
\centering
\caption{Comparison to prior art.}
\renewcommand{\arraystretch}{1.3}
\label{tab_prior}
\resizebox{\myfigwidthl}{!}{%

\begin{tabular}{  l  c  c  c  c  c c  }
\toprule
	 & Weaver \cite{weaver14} & Harpe \cite{harpe12} & Patil
                                                       \cite{patil16}
  & Liu
                                                           \cite{liu16} & \multicolumn{2}{c}{This work}
  \\
\toprule
  Technology (nm) & 90 & 90 & 28 FDSOI & 28 &
                                                  \multicolumn{2}{c}{28 FDSOI}  \\
  Fsample (MS/s) & 21 & 2 & \edit{No sampling} & 100 &  2 & 20  \\
  Core area (mm\textsuperscript{2}) & 0.18 & 0.047 & 0.0032 & 0.0047
                                                                        &
                                                                          \multicolumn{2}{
                                                                          c }{0.00312}  \\
  \midrule
  SNDR (dB) & 34.61 & 57.79 & 40 & 64.43 & 46.43 & 48.84 \\
  SFDR (dBc) & 40.81 & 72.33 & 30 &  75.42 & 61.72 & 63.11  \\
  ENOB (bits) & 5.45 & 6.7 - 9.4 &6.35 & 10.41  & 7.42 & 7.82  \\
  \midrule
  Supply (V) & 0.7 & 0.7 & 0.65 & 0.9  & 0.47 & 0.69  \\
% \edit{ Ref. pwr (\%)} & - & - & - & - &\edit{ \texttildelow 50} &
%                                                                     \edit{
%                                                                     \texttildelow
%                                                                     44}   \\
% \edit{ Other pwr (\%)} & - & - & - &- &\edit{ \texttildelow 50} &
%                                                                     \edit{
%                                                                     \texttildelow
%                                                                     56}  \\
  Pwr ($\mu$W) & 1110 & 1.64 -3.56 & 24 & 350 &  0.94 & 15.87 \\
  \midrule
  Compiled & Yes & No & No & No & \multicolumn{2}{c}{Yes}  \\
  FoM (fJ/c.step) & 838 & 2.8 - 6.6 & 3.7 & 2.6 & 2.7 & 3.5 \\
  \bottomrule
\end{tabular}
}
\end{table}



\section{Conclusion} \label{conclusion}
A low power 9-bit compiled SAR ADC has been presented with a comparator
clock generation loop that \edit{uses} the bottom plate of the CDAC.  The
ADC was compiled from text input files into a DRC/LVS clean layout and
schematic in {\technm}.
The proposed ADC achieves a FoM of 3.5 fJ/conv.step at
20 MS/s with an area of 0.00312 mm\textsuperscript{2}, and demonstrates
that a compiled SAR ADC can achieve state-of-the-art performance.

\section*{Acknowledgment}
Partial funding for this work was obtained from the Norwegian PhD
Network on Nanotechnology for Microsystems, which is sponsored by the
Research Council of Norway, Division of Science, under contract
no. 221860/F60. The authors thank L. Lewyn and H. Garvik for
technical discussions, and Nordic Semiconductor and D. Engelien-Lopes for donation of time to
perform this work.
