%\documentclass[final,11pt]{IEEEtran}
\documentclass[crop,border=0pt]{standalone}
\usepackage{times}

%%%%%%%%%%%%%%%%%%%%%%%%%%%%%%%%%%%%%%%%%%%%%%%%%%%%%%%%%%%%%%%%%%%%%% 
%% Copyright (c) 2016 Carsten Wulff Software, Norway
%% %%%%%%%%%%%%%%%%%%%%%%%%%%%%%%%%%%%%%%%%%%%%%%%%%%%%%%%%%%%%%%%%%%% 
%% Created       : wulff at 2016-11-5
%% %%%%%%%%%%%%%%%%%%%%%%%%%%%%%%%%%%%%%%%%%%%%%%%%%%%%%%%%%%%%%%%%%%% 
%% This program is free software: you can redistribute it and/or modify
%% it under the terms of the GNU General Public License as published by
%% the Free Software Foundation, either version 3 of the License, or
%% (at your option) any later version.
%% 
%% This program is distributed in the hope that it will be useful,
%% but WITHOUT ANY WARRANTY; without even the implied warranty of
%% MERCHANTABILITY or FITNESS FOR A PARTICULAR PURPOSE.  See the
%% GNU General Public License for more details.
%% 
%% You should have received a copy of the GNU General Public License
%% along with this program.  If not, see <http://www.gnu.org/licenses/>.
%%%%%%%%%%%%%%%%%%%%%%%%%%%%%%%%%%%%%%%%%%%%%%%%%%%%%%%%%%%%%%%%%%%%%% 


%\usepackage{placeins}
\usepackage{upgreek}
\usepackage{amssymb,amsmath}
\usepackage{cite}
\usepackage{verbatim}
\usepackage{listings}
\usepackage{xcolor}
\usepackage{flafter}
\usepackage{epstopdf}
\usepackage{booktabs}
\usepackage{url}
\usepackage{ textcomp }

\hyphenation{op-tical net-works semi-conduc-tor}


% -JSON styel

%\definecolor{darkGreen}{rgb}{0,0.2097,0}
%\newcommand{\myfigwidth}{\linewidth}
%\newcommand{\myfigwidtha}{\linewidth}
%\newcommand{\myfigwidthl}{\textwidth}
%\newcommand{\myfigwidthc}{0.6\linewidth}


\newcommand{\myfigwidtha}{3.3in}
\newcommand{\myfigwidthl}{7in}
\newcommand{\myfigwidthc}{2in}
%\newcommand{\myfigwidthe}{3.2in}
%\newcommand{\myfigwidthb}{5in}

\newcommand{\myfigname}{fig_}
\newcommand{\reg}[2]{{\figurename} \ref{#1}(#2)}
\newcommand{\req}[1]{{\figurename} \ref{#1}}
%\newcommand{\myeqname}{eq:}
%\newcommand{\req}[1]{(\ref{\myeqname#1})}

\definecolor{armygreen}{rgb}{0, 0.5, 0}
\newcommand{\missing}[1]{{ #1}}
\newcommand{\edit}[1]{{ #1}}
\newcommand{\secedit}[1]{{ #1}}
\newcommand{\deleted}[1]{{ #1}}
\newcommand{\moved}[1]{{ #1}}

%\newcommand{\missing}[1]{{\color{red} #1}}
%\newcommand{\edit}[1]{{\color{red} #1}}
%\newcommand{\secedit}[1]{{\color{armygreen} #1}}
%\newcommand{\deleted}[1]{{\color{violet} #1}}
%\newcommand{\moved}[1]{{\color{blue} #1}}

\newcommand{\SD}{Sigma-Delta }
\newcommand{\eqn}[1]{
  \begin{equation}
    #1
  \end{equation}}

\newcommand\JSONnumbervaluestyle{\color{blue}}
\newcommand\JSONstringvaluestyle{\color{red}}



\newif\ifcolonfoundonthisline

\makeatletter

\lstdefinestyle{json}
{
  showstringspaces    = false,
  keywords            = {false,true},
  alsoletter          = 0123456789.,
  morestring          = [s]{"}{"},
  stringstyle         = \ifcolonfoundonthisline\JSONstringvaluestyle\fi,
  MoreSelectCharTable =%
  \lst@DefSaveDef{`:}\colon@json{\processColon@json},
  basicstyle          = \fontsize{7}{7}\ttfamily,
  keywordstyle        = \fontsize{7}{7}\ttfamily\bfseries,
}

% flip the switch if a colon is found in Pmode
\newcommand\processColon@json{%
  \colon@json%
  \ifnum\lst@mode=\lst@Pmode%
  \global\colonfoundonthislinetrue%
  \fi
}

\lst@AddToHook{Output}{%
  \ifcolonfoundonthisline%
  \ifnum\lst@mode=\lst@Pmode%
  \def\lst@thestyle{\JSONnumbervaluestyle}%
  \fi
  \fi
  % override by keyword style if a keyword is detected!
  \lsthk@DetectKeywords%
}

% reset the switch at the end of line
\lst@AddToHook{EOL}%
{\global\colonfoundonthislinefalse}

\makeatother

% \lstset{  basicstyle=\tiny}

% \lstset{
% basicstyle=\fontsize{11}{13}\selectfont\ttfamily
% }


\lstset{language=Java}




%\usepackage[paperwidth=\maxdimen,paperheight=\maxdimen]{geometry}


\usepackage{tikz,pgfplots}
\usepackage{ifthen}
\usepackage{calc}
\usetikzlibrary{patterns}
%\usepackage[tightpage,active]{preview}
\usetikzlibrary{circuits.logic.US}
\usetikzlibrary{circuits.ee.IEC}
\usepgflibrary{shapes.arrows}
\usepgflibrary{arrows}
\usepackage{circuitikz}


\ctikzset{tripoles/mos style/arrows}
\ctikzset{tripoles/pmos style/emptycircle}
%\PreviewEnvironment{circuitikz}
\pagestyle{empty}
\begin{document}

\tikzstyle{cicfill}=[minimum size=1.4em]
\tikzstyle{stuff_fill}=[rectangle,fill=white,minimum size=1.4em]

\newcommand{\grayOne}{0.3 }
\newcommand{\grayTwo}{0.15 }
\newcommand{\grayThree}{0.26 }
\newcommand{\grayFour}{0.41 }
\newcommand{\grayFive}{0.61 }
\newcommand{\graySix}{0.85 }

%- Gray tone
%\definecolor{active}{rgb}{\graySix,\graySix,\graySix}
%\definecolor{poly}{rgb}{\grayFive,\grayFive,\grayFive}
%\definecolor{contact}{rgb}{\graySix,\graySix,\graySix}
%\definecolor{mOne}{rgb}{\grayFour,\grayFour,\grayFour}
%\definecolor{mTwo}{rgb}{\graySix,\graySix,\graySix}
%\definecolor{mThree}{rgb}{\grayThree,\grayThree,\grayThree}
%\definecolor{mFour}{rgb}{\grayTwo,\grayTwo,\grayTwo}

\definecolor{grey}{rgb}{1,1,1}
\definecolor{lightgrey}{rgb}{\grayFour,\grayFour,\grayFour}
\definecolor{active}{rgb}{0.4,1,0.6}
\definecolor{poly}{rgb}{1.0,0.5,0.5}
\definecolor{cut}{rgb}{0.91,0.91,0}
\definecolor{mOne}{rgb}{0.36,0.36,1}
\definecolor{mTwo}{rgb}{0.655,0.655,0}
\definecolor{mThree}{rgb}{0.3,1,1}
\definecolor{mFour}{rgb}{0,0.2097,0}





\begin{circuitikz}[circuit logic US, thick,
  transform shape,circuit ee IEC,set make contact graphic= var make contact IEC graphic] 

%  \draw[white,help lines,step=1, line width=1pt] (-2,-2) grid (20,20);

%\draw[gray] (6,4.2) rectangle (17,12.5);
\draw[lightgray] (16.7,4.5) -- ++(0,9.5);    
\draw[font=\Large] (11,14) node [anchor=south] { (a)}    ; 
\draw[lightgray] (-2.6,14) -- (24,14);    
\draw[lightgray] (5.9,4.5) -- ++(0,9.5);    

\begin{scope}[shift={(-1,17)},scale=0.64]

\newcommand*{\cicCapacitor}{
  ++(0,0) to [short,-] ++(0,-0.65)
  ++(-0.3,0) to [short,-] ++(0.6,0)
  ++(-0.6,-0.2) to [short,-] ++(0.6,0)
  ++(-0.3,0) to [short,-] ++(0,-0.65)
}

\newcommand*{\cicOTA}{
  % Start left top input
  to [short,-] ++(0.5,0)
  node [anchor=west] {$+$}
  to [short,-] ++(0,0.5)

  % Draw diagonal line from top left
  to [short,-] ++(1,-0.5)
  to [short,-] ++(1.6,0)
  node [anchor=north east] () {$P$}
  ++(-1.6,0)
  to [short,-] ++(1,-0.5)

  % Draw diagonal line from center right
  to [short,-] ++(-1,-0.5)
  to [short,-] ++(1.6,0)
  node [anchor=center,inner sep=0] (otaout) {}
  node [anchor=north east] () {$N$}
  ++(-1.6,-0)
  to [short,-] ++(-1,-0.5)

  % Draw straight line frop bottom
  to [short,-] ++(0,0.5)
  node [anchor=west] {$-$}

  % Draw comparator symbol
  ++(0.5,0.1) to [short,-] ++(0.3,0)
  to [short,-] ++(0,0.7)
  to [short,-] ++(0.3,0)
  ++(-1.1,-0.8)

  % Draw input 2 and straight line to input 1
  to [short,-] ++(-0.5,0)
  ++(0.5,0)
  to [short,-] ++(0,1)
}

\newcommand{\xsizeSarLogic}{4.2}
\newcommand{\figSarlogic}{
  \draw [black] (0,0) rectangle (3.2,3)

  % Draw enable
  (0,1) node [anchor=west] {$EI$}
  to [short,-] ++(-0.5,0)
  (3.2,1) node [anchor=east] {$EO$}
  to [short,-] ++(+0.5,0)

  % Draw clocks
  (0,2) node [anchor=west] {$CI$}
  to [short,-] ++(-0.5,0)
  (3.2,2) node [anchor=east] {$CO$}
  to [short,-] ++(0.5,0);

  % Draw CP0 capacitor, skip for x < 3
  \ifthenelse{\x > 3}{\draw
    (0.4,0) node [anchor=south] {$D_{P0}$}
    to [short,-] ++(0,-0.2)
    \cicCapacitor
    to [short,-*] ++(0,-0.1)
    ;}{}

  % Draw CP1 capacitor
  \draw
  (1.2,0) node [anchor=south] {$D_{P1}$}
  to [short,-] ++(0,-0.2)
  \cicCapacitor
  to [short,-*] ++(0,-0.1)
  (0,-1.8) to [short,-] ++(\xsizeSarLogic,0)

  % Draw CN0 capacitor
  (2.0,0) node [anchor=south] {$D_{N0}$}
  to [short,-] ++(0,-0.2)
  \cicCapacitor
  to [short,-] ++(0.1,0)
  to [short,-] ++(0,-0.2)
  to [short,-] ++(-0.1,0)
  to [short,-*] ++(0,-0.9)
  (0,-2.8) to [short,-] ++(\xsizeSarLogic,0);

  % Draw CN1 capacitor, skip for x < 3
  \ifthenelse{\x > 3}{\draw
    (2.8,0) node [anchor=south] {$D_{N1}$}
    to [short,-] ++(0,-0.2)
    \cicCapacitor
    to [short,-] ++(0.1,0)
    to [short,-] ++(0,-0.2)
    to [short,-] ++(-0.1,0)
    to [short,-*] ++(0,-0.9);
  }{}

  % Draw CK input
  \draw (0.4,3) to [short,-o] ++(0,+0.2)
  node [anchor=south] {$CK$} (1.2,3)
  to [short,-o] ++(0,+0.2)
  % Draw D output
  node [anchor=south] {$D\x$}
  ++(0,-0.2) node [anchor=north] {$D_{P1}$}

  % Draw N and P
  (2.0,3) node [anchor=north] {$P$}
  to [short,-] ++(0,1)
  (2.8,3) node [anchor=north] {$N$}
  to [short,-] ++(0,1);

  % Draw labels
  \draw (\xsizeSarLogic/2 - 1/2,1.5) node [anchor=center,font=\large]
  {\textbf{$LOGIC[\x]$}};
\begin{scope}[font=\Large]
  \draw (\xsizeSarLogic/2 - 1/2,-2.8) node [anchor=north] {$ {2^\x}
    \cdot C_{UNIT} $};
\end{scope}
}


\newcommand{\ccount}{9}

%- Start drawing

\foreach \x in {8,...,0}{
  \begin{scope}[shift={(\ccount*\xsizeSarLogic-\xsizeSarLogic  - \x*\xsizeSarLogic,0)}]
    \figSarlogic
  \end{scope}
}


% Draw OTA and connection to P and N
\draw (-0.5 + \xsizeSarLogic*\ccount , -1.8) \cicOTA {};
\draw[line width=2pt] (2,4) -- (\xsizeSarLogic*\ccount -1,4) -| (otaout);

% Draw EI and CI input
\draw (-2.0,1) node [anchor=east,font=\Large] {$\overline{CK}$} to [short,o-] ++(2,0);
\draw  (-0.6,2) node [ground,rotate=180] {};

% Draw switches
\draw (-2,-1.8) node [anchor=east,font=\Large] {$V_{P}$} to [short,o-]
++(0.5,0) 
to [make contact,name=sw1] ++(1.5,0) ;
\draw (-2,-2.8) node [anchor=east,font=\Large] {$V_{N}$} to [short,o-]
++(0.5,0) coordinate (sw1) to [make contact,name=sw2]  ++(1.5,0) ++(0.5,-0.1) node
[anchor=north east, align=center] () {Bootstrapped \\ NMOS switches};
\draw (sw1) ++(0.75,0.1) to [short,-o] ++(0,1.5) node [anchor=south,font=\Large] {$CK$};

%\draw (sw1.north) to [short, -o] ++(0,0.5)

% Draw gates from CK output
\draw (\xsizeSarLogic*\ccount,1) ++(0,-0.2) node [nor gate, point down, anchor=input 2] (nor1) {} (nor1.input 2) |- (-1 + \xsizeSarLogic*\ccount,1);
\draw (nor1.input 1) to [short,-o] ++(0,+0.5)  node [anchor=south]
{$CK$} ;

\draw (nor1.output) ++(0.3,-0.2) node [and gate,point down,anchor=input 2] (and1) {} (nor1.output) -- ++(0,-0.1) -| (and1.input 2);
\draw (and1.input 1) ++(0.3,0.3) node [not gate, point down, anchor=output] (inv2) {};
\draw (inv2.output) -- ++(0,-0.2) -| (and1.input 1);
\draw (inv2.input) -- ++(0,0.5) |- (\xsizeSarLogic*\ccount - 1,2);
\draw (and1.output) -- ++(0,-0.4) ++(0,0.2) node [anchor= west] ()
{$CK\_CMP$};
\draw (inv2.east)  ++(0.1,0) node [anchor=south west] {$X1$} ;
\draw (and1.east) ++(0.2,0.2)  node [anchor=south west] {$X2$} ;

\end{scope}


\begin{scope}[shift={(-7,6)},scale=1]
\draw[font=\Large] (8,-1) node [anchor=north] { (b)}    ;     
b
\draw

 (6,0) node [ground,rotate=-90,anchor=west] {}
    to [Tnmos, n=mn1, l=$M_{N1}$] ++(0,1.6) 
 (8,0) node [ground,rotate=-90,anchor=west] {}
    to [Tnmos, n=mn2, l=$M_{N2}$] ++(0,1.6)
    to [short,*-] ++(-2,0) node [] {}
    ++(+2,0) to [Tnmos, n=mn0, l=$M_{N0}$] ++(0,1.6)
    to [short] ++(0,1.6)
    to [Tpmos, n=mp0, l=$M_{P0}$] ++(0,1.6)
    node [rground, yscale=-1] {}
    ++(0,0.3) node [anchor=south] {$V_{\mathit{DD}}$}
(11,0) node [ground,rotate=-90,anchor=west] {}
    to [Tnmos, n=mn3, l=$M_{N3}$] ++(0,1.6) coordinate(en1)
    to [short, *-] ++(0.5,0) node [anchor=south] {$EO$} to [short,-]
    ++(0.4,0) ++(0.25,0)  node [not
    gate] (inv1) {} (inv1.output) to [short,-] ++(0.3,0) coordinate
    (eoOutput)
    (en1)
    to [Tpmos, n=mp1, l=$M_{P1}$] ++(0,1.6)
    to [Tpmos, n=mp2, l=$M_{P2}$] ++(0,1.6)
    to [Tpmos, n=mp3, l=$M_{P3}$] ++(0,1.6)
    node [rground, yscale=-1] {}
    ++(0,0.3) node [anchor=south] {$V_{\mathit{DD}}$}

(mn3.gate) to [short] ++(-1,0) to [short] ++(0,2.4) to [short,*-*]
++(-1,0)  node[short,*-] {} ++(+1,0) to [short] ++(0,2.4) to
[short] (mp3.gate)

(mn0.drain)  ++(0.1,0.3) node [anchor=west] {$A$}
(mn1.gate) to [short,o-] ++(0,0) node [anchor=east] {$P$}
(mn2.gate) to [short,o-] ++(0,0) node [anchor=east] {$N$}
(mn0.gate) to [short,o-] ++(0,0) node [anchor=east] {$EI$}
(mp1.gate) to [short,o-] ++(0,0) node [anchor=east] {$P$}
(mp2.gate) to [short,o-] ++(0,0) node [anchor=east] {$N$}
(mp0.gate) to [short,o-] ++(0,0) node [anchor=east] {$\overline{CK}$}
;


\end{scope}

\begin{scope}[shift={(8,6)},scale=1]
\draw[font=\Large] (3,-1) node [anchor=north] { (c)}    ;     
\begin{scope}
\draw
 (0,0) node [ground,rotate=-90,anchor=west] {}
    to [Tnmos, n=mn4, l=$M_{N4}$] ++(0,1.6) 
    to [Tnmos, n=mn5, l=$M_{N5}$] ++(0,1.6)
    to [Tnmos, n=mn6, l=$M_{N6}$] ++(0,1.6)
    to [Tpmos, n=mp4, l=$M_{P4}$] ++(0,1.6)
    node [rground, yscale=-1] {}
    ++(0,0.3) node [anchor=south] {$V_{\mathit{DD}}$}
(mn4.gate) to [short,o-] ++(0,0) node [anchor=east] {$P$}
(mn5.gate) coordinate(eoInput1) 
%(mn5.gate) to [short,o-] ++(0,0) node [anchor=east] {$\overline{EO}$}
(mn6.gate) to [short,o-] ++(0,0) node [anchor=east] {$EI$}
    (mp4.gate) to [short,o-] ++(0,0) node [anchor=east] {$\overline{CK}$};

    \draw (0.6,4.8) node [not gate] (inv1) {}
    ++(0.8,-0.1) node [anchor=north] {$D_{P0}$};

    \draw (inv1.output) ++(1,0) node [not gate] (inv2) {}
    ++(0.8,-0.1) node [anchor=north] {$D_{N0}$} ;

    \draw (inv1.north) -- ++(0,0.5) -- ++(0.8,0) node [rground, yscale=-1] {}
    ++(0,0.3) node [anchor=south] {$V_{\mathit{REF}}$}  ++(0,-0.3) -| (inv2.north);
    
    \draw  (0,4.8) to [short, *-] (inv1.input);
    \draw  (inv1.output) to [short, -] (inv2.input);
    \draw (inv2.output) to [short, -] ++(0.5,0) coordinate (cn0Output);
\end{scope}

\begin{scope}[xshift=5.3cm]
\draw
 (0,0) node [ground,rotate=-90,anchor=west] {}
    to [Tnmos, n=mn4, l=$M_{N7}$] ++(0,1.6) 
    to [Tnmos, n=mn5, l=$M_{N8}$] ++(0,1.6)
    to [Tnmos, n=mn6, l=$M_{N9}$] ++(0,1.6)
    to [Tpmos, n=mp4, l=$M_{P5}$] ++(0,1.6)
    node [rground, yscale=-1] {}
    ++(0,0.3) node [anchor=south] {$V_{\mathit{DD}}$}
(mn4.gate) to [short,o-] ++(0,0) node [anchor=east] {$N$}
%(mn5.gate) to [short,o-] ++(0,0) node [anchor=east] {$\overline{EO}$}
(mn5.gate) coordinate(eoInput2) 
(mn6.gate) to [short,o-] ++(0,0) node [anchor=east] {$EI$}
    (mp4.gate) to [short,o-] ++(0,0) node [anchor=east] {$\overline{CK}$};

    \draw (0.6,4.8) node [not gate] (inv1) {}
    ++(0.8,-0.1) node [anchor=north] {$D_{N1}$};

    \draw (inv1.output) ++(1,0) node [not gate] (inv2) {}
    ++(0.8,-0.1) node [anchor=north] {$D_{P1}$};
    

    \draw (inv1.north) -- ++(0,0.5) -- ++(0.8,0) node [rground, yscale=-1] {}
    ++(0,0.3) node [anchor=south] {$V_{\mathit{REF}}$}  ++(0,-0.3) -| (inv2.north);
    
    \draw  (0,4.8) to [short, *-] (inv1.input);
    \draw  (inv1.output) to [short, -] (inv2.input);
    \draw (inv2.output) to [short, -] ++(0.5,0) coordinate (cp1Output);
\end{scope}


\end{scope}



\begin{scope}[shift={(18.5,6)},scale=1]
\draw[font=\Large] (2,-1) node [anchor=north] { (d)}    ;     
%\draw[red, opacity=0.2] (-2,-1) rectangle (7,4.1);
%\draw[red,font=\large] (4,3) node [anchor=west] { This work}    ;     

\begin{scope}
\draw
 (0,0) node [ground,rotate=-90,anchor=west] {}
    to [Tnmos, n=mn4, l=$M_{N10}$] ++(0,1.6) 
    to [short,*-*] ++(2.5,0) ++(-2.5,0)
    ++(1,0) node [anchor=south] {$B$} ++(-1,-0.1) to [short,-] ++(0,2.5)
    to [Tpmos, n=mp4, l=$M_{P6}$] ++(0,1.6)
    node [rground, yscale=-1] {}
    ++(0,0.3) node [anchor=south] {$V_{\mathit{DD}}$}
 (2.5,1.5) to [short,-] ++(0,2.5) to [Tpmos, n=mp5, l=$M_{P7}$] ++(0,1.6)
    node [rground, yscale=-1] {}
    ++(0,0.3) node [anchor=south] {$V_{\mathit{DD}}$}
(mp4.gate) to [short,-] ++(0,0) coordinate(cp1Input) %node [anchor=east] {$CP1$}
(mp5.gate) to [short,-] ++(0,0) coordinate(cn0Input) %node [anchor=east] {$CN0$}
(mn4.gate) to [short,o-] ++(0,0) node [anchor=east] {$CK$};
    \draw (3,1) node [and gate] (and1) {}
    (and1.output) ++(1,-0.5) node [or gate] (or1) {}
    (and1.input 2) coordinate (eoInput3) ;%to [short,-o] ++(-0.6,0) node [anchor=east] {$\overline{EO}$};

    \draw (and1.output) -- ++(0.3,0)  |- (or1.input 1);
    \draw (2.5,1.6) |- (and1.input 1)
    (or1.input 2) to [short,-o] ++(-0.6,0) node [anchor=east] {$CI$}
    (or1.output) to [short,-o] ++(0.6,0) node [anchor=west] {$CO$};
\end{scope}

\end{scope}

\draw (eoOutput) |- (eoInput1);
\draw (eoOutput) to [short,*-] ++(0,-2.4) coordinate (eoCon) to [short,-*] ++( 5,0) |-
(eoInput2);
\draw (eoCon) -- ++(14,0) |- (eoInput3);
\draw (cp1Output) -| (cp1Input);

\draw (cn0Output) -- ++(0,2.5) -- ++(8,0) |- (cn0Input);


\end{circuitikz}


\end{document}
